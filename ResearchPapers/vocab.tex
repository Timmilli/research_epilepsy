\chapter{Vocabulary}

\section{Words}

\begin{itemize}
	\item Patients: people who have FCD (or the disease)
	\item Controls: people who don't
	\item MRI-positive: FCD (or disease) detected on the MRI by doctors
	\item MRI-negative: FCD (or disease) not detected on the MRI by doctors yet detected by AI
	\item Sensitivity: percentage of patients correctly detected
	\item Specificity: percentage of controls with zero clusters
	\item Precision: the ratio of the TP to the sum of TP and FP
	\item Recall: the ratio of TP to the sum of TP and FN
	\item EEG: electroencephalogram
	\item T1/T1-FLAIR: type of MRIs (more \href{https://mriquestions.com/t1-flair.html}{here})
	\item Heterotopia is another category of MCDs characterized by cortical cells (grey matter) encountered in inappropriate locations in the brain, as a result of interruption in their migration to the correct location in the cerebral cortex. Grey matter heterotopia may be unilateral or bilateral. Its most common form is bilateral periventricular nodular heterotopia (grey matter heterotopia lining the lateral ventricles). It can also occur in subcortical white matter (subcortical nodular heterotopia). \href{https://hal.science/tel-02062210v2/file/these.pdf}{(source)}
\end{itemize}

\section{MRI planes}

\begin{figure}[htbp]
	\centering
	\includegraphics[width=\textwidth]{"MRI_planes.jpg"}
\end{figure}
