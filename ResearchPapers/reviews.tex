\chapter{Reviews}

\section{\href{https://pmc.ncbi.nlm.nih.gov/articles/PMC3403799/}{[KABAT201204] Focal cortical dysplasia - review (Apr 2012)}}

Explaining the different types of FCDs and has some stats

\section{\href{Decrypting Cryptogenic Epilepsy: Machine Learning Methods for Detecting Cortical Malformations}{[AHMED2016] Decrypting Cryptogenic Epilepsy: Machine Learning Methods for Detecting Cortical Malformations (2016)}}

Stats on \% of MRI-positives/negatives

\section{\href{https://www.thelancet.com/article/S2352-3964(17)30470-X/fulltext}{[KIRALKORNEK2018103] Epileptic Seizure Prediction Using Big Data and Deep Learning: Toward a Mobile System (2018) }}

This gives a proof of concept about the usage of DL in epileptic anomaly detection.

However, it uses intracranial electroencephalography as images instead of MRI.

\section{\href{https://www.mdpi.com/1660-4601/18/11/5780\#B141-ijerph-18-05780}{[IJERPH18115780] Epileptic Seizures Detection Using Deep Learning Techniques A Review (2021) }}

Mostly about the study of EEG with DL, it tackles a lit bit the study of MRI with DL by providing 8 linked researches.

\section{\href{https://www.mdpi.com/1424-8220/23/16/7072}{Automatic Detection of Focal Cortical Dysplasia Using MRI: A Systematic Review (Aug 2023)}}

Review of 65 articles

\section{\href{https://www.sciencedirect.com/science/article/pii/S2405844024034297}{[ILESANMI2024e27398] Reviewing 3D convolutional neural network approaches for medical image segmentation (March 2024)}}

Not properly read yet, but lots of references and explains a lot about the methods used in 3D-CNN

\section{\href{https://www.medrxiv.org/content/10.1101/2025.04.06.25325315v1.full}{Independent Evaluation of Deep Learning Models for Detecting Focal Cortical Dysplasia (Apr 2024)}}

\section{\href{https://www.nature.com/articles/s41582-024-00965-9}{[ALFREDO20240601] Artificial intelligence in epilepsy - applications and pathways to the clinic (May 2024) }}

It tackles a lot of subjects, not only IA for epilepsy.

\section{\href{https://www.sciencedirect.com/science/article/pii/S0925231224001899}{[GANJI2024127418] Application of neuroimaging in diagnosis of focal cortical dysplasia: A survey of computational techniques (May 2024)}}

Studies 32 articles to talk about the methods used

Explains all kind of things: types of images used in FCD diagnosis, types of FCD (Type I, Type II...), and methods (Morphometric, machine learning and deep learning algorithms)

\section{\href{https://www.sciencedirect.com/science/article/pii/S0035378725004874}{[BERGER2025420] Artificial intelligence applied to epilepsy imaging: current status and future perspectives (May 2025) }}

This reviews a lot of article on this subject.
It also explains what is the procedure

It has an interesting fact about epilepsy in France that could be reused in the report or the presentation.

\section{\href{https://www.sciencedirect.com/science/article/pii/S1525505025001428}{Focal cortical dysplasia detection by artificial intelligence using MRI: A systematic review and meta-analysis (June 2025)}}

Comparing 22 studies and their methods

Lots of metrics and info on how to compare
