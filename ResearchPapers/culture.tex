\chapter{Culture} 

\section{\href{https://www.epilepsy.com/diagnosis/brain-imaging/mri}{What types of changes common in epilepsy can be found on an MRI?}}

\section{\href{https://radiologyassistant.nl/neuroradiology/epilepsy/role-of-mri}{[mri-for-epilepsy] Epilepsy - Role of MRI }}

\section{\href{https://onlinelibrary.wiley.com/doi/10.1002/acn3.51955}{Association between structural brain MRI abnormalities and epilepsy in older adults}}

\section{\href{https://www.sciencedirect.com/science/article/abs/pii/S0140673618325960}{Epilepsy in adults (2019)}}

\section{\href{https://radiopaedia.org/articles/focal-cortical-dysplasia}{[fcd] Focal cortical dysplasia (2024) }}

\section{\href{https://info-radiologie.ch/resonance_magnetique.php}{Imagerie par résonance magnétique (IRM)}}

\section{\href{https://www.nejm.org/doi/10.1056/NEJMoa1703784}{[NEJMOA1703784] Histopathological Findings in Brain Tissue Obtained during Epilepsy Surgery (2017)}}

Nice for introduction. Explaining which anomalies are the most common: 

\begin{tabular}{l|ll}
	& Children & Adults \\
	\hline
	First & Focal cortical dysplasia & Hippocampal sclerosis \\
	Second & Tumor & Tumor \\
\end{tabular}

\quote{"Overall, 57.6\% of patients with malformations of cortical development (59.9\% of children and 54.6\% of adults) were seizure-free 1 year after surgery."}

\section{\href{https://link.springer.com/article/10.1007/s00415-023-11953-2?utm_source=getftr&utm_medium=getftr&utm_campaign=getftr_pilot&getft_integrator=sciencedirect_contenthosting}{[COSTE202401] Prevalence, demographic and spatial distribution of treated epilepsy in France in 2020: a study based on the French national health data system (2023) }}

"In 2020, we identified 685,122 epilepsy cases, corresponding to an overall prevalence of 10.2 per 1000 inhabitants [95\% confidence interval 10.1–10.2], with similar rates in men and women"

\section{\href{https://www.who.int/news-room/fact-sheets/detail/epilepsy/}{[who-epilepsy] Epilepsy - WHO}}

"Around 50 million people worldwide have epilepsy, making it one of the most common neurological diseases globally"

\section{\href{https://pubmed.ncbi.nlm.nih.gov/21219302/}{[BLUMCKE201101] The clinicopathologic spectrum of focal cortical dysplasias: a consensus classification proposed by an ad hoc Task Force of the ILAE Diagnostic Methods Commission (Jan 2011)}}

Full article: \href{https://www.ilae.org/files/dmfile/The-clinicopathologic-spectrum-of-FCD-Blumcke-2011.pdf}{here}

ILAE article about FCD classification

\section{\href{https://academic.oup.com/brain/article/130/2/574/287391}{[10.1093/brain/awl364] Surgical outcome and prognostic factors of frontal lobe epilepsy surgery (Jan 2007)}}

About the importance of the correct identification of FCD before surgery to boot post-surgery outcome

\section{\href{https://jamanetwork.com/journals/jamaneurology/fullarticle/798793}{[10.1001/archneurol.2009.283] Characteristics and Surgical Outcomes of Patients With Refractory Magnetic Resonance Imaging–Negative Epilepsies (Dec 2009)}}

About the importance of the correct identification of FCD before surgery to boot post-surgery outcome

\section{\href{https://case.edu/med/neurology/NR/MRI%20Basics.htm}{ Magnetic Resonance Imaging (MRI) of the Brain and Spine: Basics}}

Explains the differences between each parameters of an MRI and cool images
