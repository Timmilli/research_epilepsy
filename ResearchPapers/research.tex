\chapter{Researches}

\section{Reviews}

\subsection{\href{https://www.mdpi.com/1660-4601/18/11/5780\#B141-ijerph-18-05780}{Epileptic Seizures Detection Using Deep Learning Techniques A Review (2021)}}

Mostly about the study of EEG with DL, it tackles a lit bit the study of MRI with DL by providing 8 linked researches.

\subsection{\href{https://www.nature.com/articles/s41582-024-00965-9}{Artificial intelligence in epilepsy - applications and pathways to the clinic (May 2024)}}

It tackles a lot of subjects, not only IA for epilepsy.

\subsection{\href{https://www.sciencedirect.com/science/article/pii/S0035378725004874}{Artificial intelligence applied to epilepsy imaging: current status and future perspectives (May 2025)}}

This reviews a lot of article on this subject.
It also explains what is the procedure

It has an interesting fact about epilepsy in France that could be reused in the report or the presentation.

\section{Articles}

\subsection{\href{https://www.sciencedirect.com/science/article/pii/S187887502032698X\#abssec0015}{Convolutional Neural Networks for Pediatric Refractory Epilepsy Classification Using Resting-State Functional Magnetic Resonance Imaging (May 2021)}}
\label{sub:sec:cnn}

Study made to evaluate the performance of CNN in "the classification of patients with paediatric epilepsy from healthy control"

To modify CNNs, some parameters were manually changed, such as: 3 hyperparameters, learning rate, dropout rate and regularization, and number of epoch.

To evaluate them ($\sim$4000 models), the model with the highest sensitivity was kept.

\subsubsection{Methods}

Full explanations of each step in the paper, figure summing up here: \citeref{fig:cnn}

\begin{figure}[htbp]
	\centering
	\includegraphics[width=\textwidth]{"CNN_explanation.jpg"}
	\caption{Linked to this research: \citeref{sub:sec:cnn}}%
	\label{fig:cnn}
\end{figure}

\subsubsection{Results}

Results here: \citeref{tab:cnn}

\begin{table}[htbp]
	\centering
	\fbox{
	\begin{tabular}{l c}
		n & 322 \\
		patients & 63 \\
		controls & 259 \\
		train ratio & 0.6 \\
		validate ratio & 0.2 \\
		test ratio & 0.2 \\
	\end{tabular}
}
	\caption{Benchmark}

	\fbox{
	\begin{tabular}{l c}
		Sensitivity & 85\% \\
		Specificity & 71\% \\
		F1 score & 0.56 \\
	\end{tabular}
}
	\caption{Results of \citeref{sub:sec:cnn}}%

	\fbox{
		\begin{tabular}{l c}
			Dropout rate & 50\% \\
			Learning rate & $1e-4$ \\
			Epoch & 181 \\
		\end{tabular}
	}
	\caption{Best model parameter values}
	\label{tab:cnn}
\end{table}

\subsection{\href{https://academic.oup.com/brain/article/145/11/3859/6659752?login=true}{Interpretable surface-based detection of focal cortical dysplasias: a Multi-centre Epilepsy Lesion Detection study (August 2022)}}
\label{sub:sec:meld_research}

Code via GitHub here: \citeref{sub:sec:meld_github}

\subsubsection{Methods}

Explanation here: \citeref{fig:meld_explanation}

Cohort splitting: 50-50 and 10-fold cross-validation

From MRI images, brain is reconstructed via \emph{FreeSurfer} library. It then consists in a lot of vertices and edges.
These are analysed by the model to declare lesioned vertices.

\begin{figure}[htbp]
	\centering
	\includegraphics[width=\textwidth]{"meld_method.pdf"}
	\caption{Linked to this research: \citeref{sub:sec:meld_research}}%
	\label{fig:meld_explanation}
\end{figure}

\subsubsection{Results}

\begin{table}[htbp]
	\centering
	\fbox{
	\begin{tabular}{l c}
		n & 1015 \\
		patients & 618 \\
		controls & 397 \\
		train ratio & 0.5 \\
		test ratio & 0.5 \\
	\end{tabular}
}
	\caption{Benchmark}

	\fbox{
	\begin{tabular}{l c}
		Sensitivity & 59\% \\
		Specificity & 54\% \\
	\end{tabular}
}
	\caption{Results}
\end{table}

\subsection{\href{https://www.sciencedirect.com/science/article/pii/S1525505015002322}{Cortical feature analysis and machine learning improves detection of “MRI-negative” focal cortical dysplasia (2015)}}
\label{sub:sec:res3}

From MRI images, brain is reconstructed via \emph{FreeSurfer} library. It then consists in a lot of vertices and edges.
These are analysed by the model to declare lesioned vertices.

\subsubsection{Results}

Here: \citeref{tab:res3}

\begin{table}[htbp]
	\centering
	\fbox{
	\begin{tabular}{l c}
		n & 93 \\
		patients & 31 \\
		controls & 62 \\
		train ratio & 0.5 \\
		test ratio & 0.5 \\
	\end{tabular}
}
	\caption{Benchmark}

	\fbox{
	\begin{tabular}{l c}
		not clear but & \\
		Sensitivity & 86\% \\
		Specificity & 58\% \\
	\end{tabular}
}
	\caption{Results from \citeref{sub:sec:res3}}%
	\label{tab:res3}
\end{table}

\subsection{\href{https://www.sciencedirect.com/science/article/pii/S1746809419301211\#sec0010}{Automatic detection and localization of Focal Cortical Dysplasia lesions in MRI using fully convolutional neural network (2019)}}
\label{sub:sec:res4}

Some maths for the loss function, clear explanation on how it works.
Lots of information about the whole setup, procedure, method, neural network\dots

MRI images are taken in the sagittal plane

\subsubsection{Preprocessing}

\begin{figure}[htbp]
	\centering
	\includegraphics[width=\textwidth]{"MRI_example.jpg"}
	\caption{From research: \citeref{sub:sec:res4}}
	\label{fig:res4:1}
\end{figure}

\begin{figure}[htbp]
	\centering
	\includegraphics[width=\textwidth]{"MRI_example_2.jpg"}
	\caption{From research: \citeref{sub:sec:res4}}
	\label{fig:res4:2}
\end{figure}

\subsubsection{Method}

Fully connected convolutional network is used in order to keep the size of the image and to result into an image where each pixel corresponds to a probability of being lesioned.

Loss function: loss = Binary Cross-Entropy + Dice Loss

5-fold cross-validation where one patient frames would stay in the same fold to avoid biases

\begin{table}
	\centering
	\fbox{
	\begin{tabular}{l c}
		epochs & 100 \\
		learning rate & 0.4 \\
	\end{tabular}
}
	\caption{Interesting values of \citeref{sub:sec:res4}}
\end{table}

\begin{figure}[htbp]
	\centering
	\includegraphics[width=\textwidth]{"CNN_explanation_2.jpg"}
	\caption{From research: \citeref{sub:sec:res4}}
	\label{fig:res4:3}
\end{figure}

\subsubsection{Results}

Here: \citeref{tab:res4}

\begin{table}[htbp]
	\centering
	\fbox{
	\begin{tabular}{l c}
		subjects & 43 \\
		n (MRI frames) & 8849 \\
		patients & 1431 \\
		controls & 7418 \\
		train ratio & 0.65 \\
		test ratio & 0.18 \\
		validation ratio & 0.17 \\
	\end{tabular}
}
	\caption{Benchmark}

	\fbox{
	\begin{tabular}{l c}
		Recall & 40.10 \% \\
		Precision & 80.69 \% \\
		Dice-coefficient & 52.47 \\
	\end{tabular}
}
	\caption{Results from: \citeref{sub:sec:res4}}
	\label{tab:res4}
\end{table}

\subsection{\href{https://link.springer.com/chapter/10.1007/978-3-030-00931-1_56}{Deep Convolutional Networks for Automated Detection of Epileptogenic Brain Malformations (2018)}}
\label{sub:sec:res5}

This one uses "two identical CNN which weights are optimized independently".

MRI images used are taken in the transverse plane

\subsubsection{Method}

5-fold corss-validation

\begin{figure}[htbp]
	\centering
	\includegraphics[width=\textwidth]{"two_CNN_explanation.png"}
	\caption{Top panel: Convolutional network architecture (CNNx) for two-label (lesional vs. non-lesional) classification. Bottom panel: Training and testing schema using two-stage CNNx cascade (CNN1/CNN2). From research in \citeref{sub:sec:res5}}
\end{figure}

\subsubsection{Results}

Here: \citeref{tab:res5}

\begin{table}[htbp]
	\centering
	\fbox{
		\begin{tabular}{l c}
			not clear & \\
			n &  \\
			patients & 107 \\
			controls & 101 \\
			train ratio & 0.75 \\
			test ratio & 0.25 \\
		\end{tabular}
	}
	\caption{Benchmark}

	\fbox{
		\begin{tabular}{l c}
			Sensitivity & 87\% \\
			Specificity & 95\% \\
		\end{tabular}
	}
	\caption{Results from \citeref{sub:sec:res5}}
	\label{tab:res5}
\end{table}

\subsection{\href{https://www.sciencedirect.com/science/article/pii/S0920121121002709\#sec0010}{Deep learning-based diagnosis of temporal lobe epilepsy associated with hippocampal sclerosis: An MRI study (2021)}}
\label{sub:sec:res6}

Not to actually detect FCD but works same.

Lots of details on patient and control image acquisitions.
Also on the detail of MRIs

MRI images are taken in the transverse plane

\subsubsection{Method}

5-fold cross-validation, 
use of CNN, 
data augmentation (shifting height and width, zooming, shearing, flipping),
loss function: RMSProp (from \href{https://www.sciencedirect.com/science/article/pii/S0920121121002709\#bib21}{here}),
code nor benchmark given,

\begin{figure}[htbp]
	\centering
	\includegraphics[width=\textwidth]{"CNN_explanation_3.jpg"}
	\caption{MTLE-HS means patient, Normal means control. From \citeref{sub:sec:res6}}
\end{figure}

\subsubsection{Results}

Here: \citeref{tab:res6}

\begin{table}[htbp]
	\centering
	\fbox{
	\begin{tabular}{l c}
		n & 141 \\
		patients & 85 \\
		controls & 56 \\
		train ratio & 0.8 \\
		test ratio & 0.2 \\
	\end{tabular}
	}
	\caption{Benchmark}

	\fbox{
	\begin{tabular}{l c}
		Accuracy & 87.8\% \\
		Sensitivity & 91.1\% \\
		Specificity & 83.5\% \\
	\end{tabular}
	}
	\caption{Results from \citeref{sub:sec:res6}}
	\label{tab:res6}
\end{table}

\subsection{\href{https://academic.oup.com/brain/advance-article/doi/10.1093/brain/awaf020/7972755\#510821169}{Redefining diagnostic lesional status in temporal lobe epilepsy with artificial intelligence (May 2025)}}
\label{sub:sec:3dcnn_article}

\subsubsection{Model \& Method}

3D CNN,
10 runs of 5-fold cross-validation,
no biases cuz of group imbalance,
training:test → 80:20
training:validation → 80:20

\begin{table}[htbp]
	\centering
	\fbox{
	\begin{tabular}{l c}
		participants & 1178 \\
		n & 1178 \\
		patients & 589 \\
		controls & 589 \\
		train ratio & 0.8 \\
		test ratio & 0.2 \\
	\end{tabular}
	}
	\caption{Benchmark}

	\fbox{
	\begin{tabular}{l c}
		Accuracy & 85.2\% \\
		Sensitivity & 80.5\% \\
		Specificity & 89.8\% \\
	\end{tabular}
	}
	\caption{Results}
\end{table}
